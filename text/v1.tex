
U-I Kennlinie eines ohmschen Widerstandes ( Linarer Widerstand)

\begin{figure}[h!]
  \begin{center}
  \begin{tikzpicture}
    % horizontal axis
    \draw[->] (0,0) -- (4,0) node[anchor=north] {$I$};
    \draw[->] (0,0) -- (0,3) node[anchor=west] {$U$};
    % ranges
   \draw (0,0) -- (3,3);
  \end{tikzpicture}
\end{center}
\end{figure}
$I=(U) = G * U $\\
$U(I) = R * I $\\
$ R = m =\frac{\Delta U}{\Delta I} $ (m ist die Steigung der gerade)\\
\begin{figure}[h!]
  \begin{center}
  Anmerkung: Grob
    \begin{tikzpicture}
    % horizontal axis
    \draw[->] (-2,0) -- (4,0) node[anchor=north] {$I$};
    \draw[->] (0,-2) -- (0,3) node[anchor=west] {$U$};
    % ranges
   \draw (-2,-0.1) -- (0,0) cos (1.5,3);
  \end{tikzpicture}
  \end{center}
\end{figure}

$y=\rho$

$R_2 = R_1 + \Delta R$

$\alpha $ ists hier Temperaturwerte 
$\Delta R = R_1 \dot \alpha \dot \Delta \theta$
$R_2 R_1 (1 + \alpha \cdot \Delta \theta) $
Notiz: welches delte ist nur hier gemeint?

Widerstandswert bliebt nicht konstant

Bei jedem Arbeitspunkt $R = \frac{U}{I}$\\
$ P_1: R_1 = \frac{U_1}{I_1} $\\
$ P_2: R_2 = \frac{U_2}{I_1} $\\

$ R_1 != R_2 $\\

Hier ist $R != \frac{\Delta u}{\Delta I} $\\

Es ist nicht mehr Proportional

Keine proportional Abhänigkeit vorhanden!
$R$ ist nicht konstant.

\section{Heiß und Klar Leiter}

Schlagwort Überwacher

Nicht linare Widerstände
NTC -  Negative Temprature Coefficient 

\begin{figure}[h!]
  \begin{center}
  \begin{tikzpicture}
    % horizontal axis
    \draw[->] (0,0) -- (4,0) node[anchor=north]{$\upsilon / ^{\circ}\mathrm{C}$};
    \draw[->] (0,0) -- (0,3) node[anchor=west] {$R/\Omega$};
    % ranges
   \draw (0,1.6) sin (4,1);
  \end{tikzpicture}
\end{center}
\end{figure}
\begin{figure}[h!]
  \begin{center}
     Bauzeichen\\
    \begin{circuitikz}[european]
    \draw (0,0) to[thRn] (5,0);
    \end{circuitikz}
  \end{center}
\end{figure}

PTC  - Positive Temperature Coefficient
\begin{figure}[h!]
  \begin{center}
  Anmerkung: Grobe Darstellung
  \begin{tikzpicture}
    % horizontal axis
    \draw[->] (0,0) -- (7,0) node[anchor=north]{$T/ ^{\circ}\mathrm{C}$};
    \draw[->] (0,0) -- (0,5) node[anchor=west] {$R/\Omega$};
    % ranges
   \draw (0,1.4) cos (2,1) cos(3,3) sin (4,5) -- (5,5.1);

   \draw[dotted] (2,1) -- (2,0) node[anchor=south]{$T_A$}; %
   \draw[dotted] (2.4,1.4) -- (2.4,0)node[anchor=south]{$T_N$};
   \draw[dotted] (3.9,4.9) -- (3.9,0)node[anchor=south]{$T_E$};

   \draw[dotted] (0,1) -- (6,1)node[anchor=west]{$R_A$};
   \draw[dotted] (0,1.4) -- (6,1.4)node[anchor=west]{$R_N$};
  \end{tikzpicture}
\end{center}
\end{figure}\\
Wir kleiner bis $T_A$\\
Einsatzbereich: \\
 $T_N - T_E $\\
\\
A: Anfrangspunkt\\
$T_A - T_N$: Nichtlinarer Breich\\
\\
$T_N - T_E$: Gültigkeitsbereich\\


\begin{figure}[h!]
  \begin{center}
     Bauzeichen\\
    \begin{circuitikz}[european]
    \draw (0,0) to[thRp] (5,0);
    \end{circuitikz}
  \end{center}
\end{figure}

NTC und PTC werden in zwei verschiedenen Gruppen eingeteilt:
1. Fremdrwärmte
2. Eigenerwärmte 

\subsection{Varistoren - VDR}
Spannungsunabhänige Widerstände

\begin{figure}[h!]
  \begin{center}
  Grober verlauf
  \begin{tikzpicture}
    % horizontal axis
    \draw[->] (0,0) -- (7,0) node[anchor=north]{$U/V$};
    \draw[->] (0,0) -- (0,4) node[anchor=west] {$R/\Omega$};
    % ranges
   \draw (1,3) sin (7,1.5);
  \end{tikzpicture}
\end{center}
\end{figure}

U-I 

\begin{figure}[h!]
  \begin{center}
  Grober verlauf
  \begin{tikzpicture}
    % horizontal axis
    \draw[->] (-4,0) -- (4,0) node[anchor=north]{$U/V$};
    \draw[->] (0,-4) -- (0,4) node[anchor=west] {$R/\Omega$};
    % ranges
   \draw (-2,-3) sin (0,0) cos (2,4);
  \end{tikzpicture}
\end{center}
\end{figure}
Voltage Dependent Resistor 

Mit den Varistor werden $C$-und $\beta$-werte bekannt gegeben.
$U = C \cdot I^{\beta}$\\
$R = \frac{U}{I} = C * \frac{I^{\beta}}{I} $

C: Konstante
Gibt die Spannung an, bei der ein Storm $1A$ durch den VDR fließt. 

$C = 15 - 5.000$

$\beta$: Regelfaktor: ein Maß fuer die Steilheit der Kennlinine

$beta = 0,15 - 0,4 $

$U = C * I^{\beta}$
$\frac{U}{C} = I^{\beta} | \frac{1}{\beta}$
$(\frac{U}{C})^{\frac{1}{\beta}} = (I^{beta})^{\frac{1}{\beta}}$
$(\frac{U}{C}) = I \rightarrow R = \frac{U}{(\frac{U}{C})^{\frac{1}{\beta}}}$
 
\newpage
\subsection{Fotowiderstand} % (fold)
\label{sub:fotowiderstand}
Fotowiderstand - LDR Light Depende Resistor

\begin{figure}[h!]
  \begin{center}
     Bauzeichen\\
    \begin{circuitikz}[european]
    \draw (0,0) to[phR] (5,0);
    \end{circuitikz}
  \end{center}
\end{figure}
% subsection f (end)
